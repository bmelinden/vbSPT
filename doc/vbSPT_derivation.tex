We now derive the core elements of the variational algorithm for
single trajectories, following earlier Variational Bayesian/Ensemble
learning treatments of HMMs\cite{Bronson2010,Mackay1997,Beal2003}.  To
make it easier to choose prior distributions however, we
reparameterize the transition matrix to separate out dwell time
distributions, and write
\begin{equation}\label{eq:aBdef}
A_{ij}=\delta_{ij}(1-a_i)+(1-\delta_{ij})a_iB_{ij}=
(1-a_i)^{\delta_{ij}}a_i^{1-\delta_{ij}}B_{ij}^{1-\delta_{ij}},
\end{equation}
with constraints
\begin{equation}
0\le a_j\le 1,\quad B_{ii}=0,\quad \sum_{j\ne i}B_{ij}=1.
\end{equation}

To start with, we will combine the model
Eqs. (\ref{eq:efactor1}-\ref{xmodel1}) with the new parameterization,
\begin{multline}
\label{eq:efactor1b}
p(\x_{1:T},s_{1:T-1},\vec{\gamma},\matris{A},\vec\pi|N)\\
=p(\x_{1:T}|s_{1:T-1},\vec{\gamma})
p(s_{1:T-1}|\matris{B},\vec{a},\vec\pi)
p(\vec{\gamma}|N)p(\matris{B}|N)p(\vec{a}|N)p(\vec\pi|N),
\end{multline}
\begin{align}
\label{smodel1b}
  p(s_{1:T-1}|\matris{B},\vec a,\vec{\pi})=&
  \prod_{m=1}^N\pi_m^{\delta_{m,s_1}}
  \prod_{t=1}^{T-1}\prod_{k,j=1}^N 
  \left((1-a_k)^{\delta_{kj}}a_k^{1-\delta_{kj}}
  B_{kj}^{1-\delta_{kj}}\right)^{\delta_{k,s_t}\delta_{j,s_{t+1}}},\\
    \label{xmodel1b}
    p(\x_{1:T}|s_{1:T-1},\vec\gamma)=&
    \prod_{t=1}^{T-1}\prod_{k=1}^N\left(
    \frac{\gamma_k}{\pi}\right)^{\frac d2\delta_{k,s_t}}
    e^{-\delta_{k,s_t}\gamma_k(\x_{t+1}-\x_t)^2}. 
\end{align}
and feed it into the mean-field machinery of Eqs.~\eqref{eq:update1}
and \eqref{eq:update2} (with
$\theta=\vec{\gamma},\matris{B},\vec{a},\vec{\pi}$,),
\begin{align}
  \ln q(\theta)=&-\ln Z_\theta+\ln p(\theta|N)
  +\mean{\ln p(x,s|\theta,N)}_{q(s)},\tag{\ref{eq:update1}}\\
  \ln q(s)=&-\ln Z_s+\mean{\ln p(x,s|\theta,N)}_{q(\theta)}.
\tag{\ref{eq:update2}}
\end{align}
which also includes choosing functional forms for the prior
distributions that make the averages tractable. In the next
subsections, we first derive the parameter distributions and the
update rule that constitute the M-step of the EM iterations, then go on
to the distribution of hidden states and its update rule (the E-step),
and finally describe how to compute the lower bound $F$ after a
completed E-step.

\subsection{Parameters}
We have four types of parameters,
$\theta=(\vec\pi,\matris{B},\vec a,\vec \gamma)$, and it will turn out that
if we choose priors that factorize in a clever way,
\begin{equation}\label{pfactorization}
p(\theta|N)=p(\vec{\gamma},\vec{\pi},\matris{A}|N)
=p(\vec\pi|N)\prod_jp(\gamma_j|N)p(B_{j,:}|N)p(a_j|N),
\end{equation}
where $B_{j,:}$ means row $j$ of $\matris{B}$, then the variational
distributions factorize in the same way, which simplifies the
computations. Hence, if we substitute factorized priors of this form,
plus Eqs.~\eqref{smodel1b} and \eqref{xmodel1b}, into the parameter
distribution equation, Eq.~\eqref{eq:update1}, we get the following
variational distribution for the parameters:
\begin{multline}\label{Maverage}
  \ln q(\vec{\pi},\matris{B},\vec a,\vec{\gamma})=
  -\ln Z_\theta  
    +\ln p(\matris{B}|N)+\ln p(\vec{a}|N)+\ln p(\vec{\pi}|N)\\
    +\mean{\ln p(s_{1:T-1}|\matris{B},\vec a,\vec{\pi)}}_{q(s_{1:T-1})}
  +\ln p(\gamma|N)+\mean{\ln p(\x_{1:T}|s_{1:T-1},\vec{\gamma})}_{q(s_{1:T-1})}\\
  =-\ln Z_\theta 
  +\sum_{j=1}^N\Big[\mean{\delta_{j,s_1}}_{q(s_{1:T-1})}\ln\pi_j+\ln p(\pi_j)\Big]\\
  +\sum_{j=1}^N\sum_{t=1}^{T-2}\bigg(
  \ln p(B_{j,:})+\sum_{k=1}^N(1-\delta_{jk})
  \mean{\delta_{j,s_t}\delta_{k,s_{t+1}}}_{q(s_{1:T-1})}\ln B_{jk}
    \bigg)\\
  +\sum_{j=1}^N\sum_{t=1}^{T-2}\bigg(\ln p(a_j)
  +\sum_{k=1}^N\mean{\delta_{j,s_t}\delta_{k,s_{t+1}}}_{q(s_{1:T-1})}
  \Big((1-\delta_{jk})\ln a_j+\delta_{jk}\ln (1-a_j)\Big)
    \bigg)\\%  -\ln p(\x_1)\\
  +\sum_{j=1}^N\Bigg[\sum_{t=1}^{T-1}
  \bigg(
  \frac d2\mean{\delta_{j,s_t}}_{q(s_{1:T-1})}\ln\frac{\gamma_j}{\pi}
  -\mean{\delta_{j,s_t}}_{q(s_{1:T-1})}(\x_{t+1}-\x_t)^2\gamma_j
  \bigg)+\ln p(\gamma_j)\Bigg].
\end{multline}
From this, we can read out the functional form of the variational
distribution, which factorize in the same was as the prior. We choose
conjugate priors\cite{Mackay1997,Beal2003}, which enables a simple
interpretation of the prior distribution in terms of fictitious
'pseudo'-observations.

\paragraph{Initial state and transition rates:}
Following earlier
work\cite{Mackay1997,Beal2003,Bronson2009,Persson2013}, we choose
Dirichlet (a multivariate version of the beta
distribution\cite{wiki:dirichlet}) priors for the initial state
distribution and for each row of the transition matrix $\matris{B}$,
and beta distributions for the elements of $\vec a$. This makes the
variational distributions Dirichlet and beta distributions as well
(they are their own conjugates). The Dirichlet density function, in
this case for $\vec\pi$, is
\begin{equation}\label{eq:p0pi}
        q(\vec\pi)=\Dir(\vec\pi|\vec{w}^{(\vec{\pi})})=\frac{1}{B(\vec
        w^{(\vec\pi)})}\prod_j \pi_j^{(w_j^{(\vec\pi)}-1)},
\end{equation}
with the constraints $0\le \pi_j\le1$ and $\sum_j\pi_j=1$, and
normalization constant $B(\vec
w^{(\vec\pi)})=\prod_j\Gamma(w_j^{(\vec\pi)})/\Gamma(\sum_kw_k^{(\vec\pi)})$\cite{wiki:dirichlet},
and the beta distribution is the special case of two components,
\begin{equation}
\beta(x|u,v)=\frac{\Gamma(u+v)}{\Gamma(u)\Gamma(v)}x^{u-1}(1-x)^{v-1}.
\end{equation}
Inspection of Eq.~\eqref{Maverage} reveals that
\begin{equation}\label{VBM_pi}
  q(\vec{\pi})=\Dir(\vec{\pi}|\vec{w}^{(\vec{\pi})}),\quad
w_j^{(\vec{\pi})}=\tilde
w_j^{(\vec{\pi})}+\mean{\delta_{j,s_1}}_{q(s)},
\end{equation}
\begin{equation}\label{VBM_B}
  q(\matris{B})=\prod_j
  \Dir(B_{j,:}|w_{j,:}^{(\matris{B})}),\quad
w_{jk}^{(\matris{B})}=\tilde w_{jk}^{(\matris{B})}
  +\sum_{t=1}^{T-2}\mean{\delta_{j,s_t}\delta_{k,s_{t+1}}}_{q(s)}\text{, ($k\ne j$),}
\end{equation}
\begin{align}
  q(\vec{a})=&\prod_j \beta(a_j|w_{j1}^{(\vec{a})},w_{j2}^{(\vec{a})}),&
  w_{j1}^{(\vec{a})}=&\tilde w_{j1}^{(\vec{a})}
  +\sum_{t=1}^{T-2}\mean{\delta_{j,s_t}(1-\delta_{j,s_{t+1}})}_{q(s)},\nonumber\\
  &&  
w_{j2}^{(\vec{a})}=&\tilde w_{j2}^{(\vec{a})}
  +\sum_{t=1}^{T-2}\mean{\delta_{j,s_t}\delta_{j,s_{t+1}}}_{q(s)},
\label{VBM_a}
\end{align}
with where $\tilde w_j^{(\vec{\pi})}$, $\tilde w_{jk}^{(\matris{B})}$,
and $\tilde w_{jk}^{(\vec{a})}$ are pseudocounts in the prior
distributions.  The total number of pseudocounts (for each
distribution) is called the prior strength. The following
average\cite{Beal2003} and mode values will be needed:
\begin{align}
  \mean{\ln \pi_i}_{q(\vec{\pi})}=&\psi(w_i^{(\vec{\pi})})-\psi(w_0^{(\vec{\pi})}),
  &w_0^{(\vec{\pi})}=&\sum_{i=1}^Nw_i^{(\vec{\pi})},\label{lnpiaverage}\\
  \mean{\ln a_j}_{q(\vec{a})}=&\psi(w_{j1}^{(\vec{a})})-\psi(w_{j0}^{(\vec{a})}),&
  w_{j0}^{(\vec{a})}=&w_{j1}^{(\vec{a})}+w_{j2}^{(\vec{a})},\\
  \mean{\ln (1-a_j)}_{q(\vec{a})}=&\psi(w_{j2}^{(\vec{a})})-\psi(w_{j0}^{(\vec{a})}),\\
  \mean{\ln B_{jk}}_{q(\matris{B})}=&\psi(w_{jk}^{(\matris{B})})
    -\psi(w_{j0}^{(\matris{B})}),&
  w_{j0}^{(\matris{B})}=&\sum_{k\ne j}w_{jk}^{(\matris{B})},
\end{align}
where $\psi$ is the digamma function.  Some more expectation
($\mean{\cdot}_{\ldots}$) and mode ($^*_{\ldots}$) values will be
useful for parameter inference:
\begin{align}
  \pi^*_{i\;q(\vec\pi)}=&\frac{w_i^{(\vec\pi)}-1}{w_0^{(\vec\pi)}-N}, &
  \mean{\pi_i}_{q(\vec{\pi})}=&\frac{w_i^{(\vec\pi)}}{w_0^{(\vec\pi)}}, \\
  \Var[\pi_i]_{q(\vec{\pi})}=&
  \frac{w_i^{(\vec\pi)}(w_0^{(\vec\pi)}-w_i^{(\vec\pi)})}{
    (w_0^{(\vec\pi)})^2(w_0^{(\vec\pi)}+1)},\\
  a^*_{i\;q(\vec a)}=&\frac{w_{i1}^{(\vec{a})}-1}{w_{i0}^{(\vec{a})}-2}, &
  \mean{a_i}_{q(\vec{a})}=&\frac{w_{i1}^{(\vec{a})}}{w_{i0}^{(\vec{a})}},\\
  (1-a_i)^*_{q(\vec{a})}=&\frac{w_{i2}^{(\vec{a})}-1}{w_{i0}^{(\vec{a})}-2}, &
  \mean{1-a_i}_{q(\vec{a})}=&\frac{w_{i2}^{(\vec{a})}}{w_{i0}^{(\vec{a})}},\\
  \Var[a_j]=&\frac{w_{j1}^{(a)}w_{j2}^{(a)}}{(w_{j0}^{(a)})^2(1+w_{j1}^{(a)})},&
\Var[1-a_j]=&\Var[a_j]\\
  B^*_{jk\;q(\matris{B})}=&\frac{w_{jk}^{(\matris{B})}-1}{w_{j0}^{(\matris{B})}-N+1}&
  \mean{B_{jk}}_{q(\matris{B})}=&\frac{w_{jk}^{(\matris{B})}}{w_{j0}^{(\matris{B})}},\\
\Var[B_{jk}]=&
\frac{w_{jk}^{(\matris{B})}(w_{j0}^{(\matris{B})}-w_{jk}^{(\matris{B})})}{
(w_{j0}^{(\matris{B})})^2(1+w_{j0}^{(\matris{B})})},&
\mean{B_{jk}^2}=&\frac{w_{jk}^{(\matris{B})}(1+w_{jk}^{(\matris{B})})}{
        w_{j0}^{(\matris{B})}(1+w_{j0}^{(\matris{B})})},\\
\mean{A_{jj}}_{q(\vec{a})q(\matris{B})}=&
\mean{1-a_j}=\frac{w_{j2}^{(\vec{a})}}{w_{j0}^{(\vec{a})}},&
\mean{A_{jk}}_{q(\vec{a})q(\matris{B})}=&
\frac{w_{j1}^{(\vec{a})}w_{jk}^{(\matris{B})}}{w_{j0}^{(\vec{a})}w_{j0}^{(\matris{B})}}.
\end{align}
\begin{align}
\Var[A_{jj}]=&\Var[1-a_j]=\Var[a_j],\\
\Var[A_{jk}]=&\mean{a_j^2B_{jk}^2}-\mean{a_jB_{jk}}^2
        =\Var[a_j]\mean{B_{jk}^2}+\mean{a_j}^2\Var[B_{jk}].
\end{align}
Note that the transition counts
for $\vec a$ and $\matris{B}$ are related via
\begin{equation}
w_{j0}^{(\matris{B})}=w_{j2}^{(\vec{a})}-\tilde w_{j2}^{(\vec{a})}+\sum_{k\ne j}\tilde w_{jk}^{(\matris{B})},
\end{equation}
and would be equal (and moments of $A_{jk}$ simpler) if the priors
were chosen such that $\sum_{k\ne j}\tilde
w_{jk}^{(\matris{B})}=\tilde w_{j1}^{(\vec{a})}$. The point of
reparameterizing $A_{ij}$ however, is to be able to choose priors such
as this is not the case. Finally, mean dwell times (in units of
$\Delta t$) is $\tau_j=a_j^{-1}$. This gives the variational density
function
\begin{equation}
q(\tau_j)=q(a_j(\tau_j))\left|\frac{da_j}{d\tau_j}\right|
=\frac{\Gamma(w_{j1}^{(\vec{a})})\Gamma(w_{j2}^{(\vec{a})})}{\Gamma(w_{j0}^{(\vec{a})})}
\tau_j^{-w_{j0}^{(\vec{a})}}(\tau_j-1)^{w_{j2}^{(\vec{a})}-1},
\quad \tau_j\ge 1,
\end{equation}
which means that
\begin{align}
\mean{\tau_j}=&\mean{a_j^{-1}}=
\frac{w_{j0}^{(\vec{a})}}{w_{j1}^{(\vec{a})}}=\frac{1}{\mean{a_j}},\\
\tau_j^*=&\frac{w_{j0}^{(\vec{a})}}{1+w_{j1}^{(\vec{a})}},\\
\mean{\tau_j^2}=&\mean{a_j^{-2}}_{q(\vec{a})}=
\mean{\tau_j}\frac{w_{j0}^{(\vec{a})}-1}{w_{j1}^{(\vec{a})}-1}
=\mean{\tau_j}^2\frac{w_{j0}^{(\vec{a})}-1}{w_{j0}^{(\vec{a})}-\mean{\tau_j}},\\
\Var(\tau_j)=&
\mean{\tau_j^2}-\mean{\tau_j}^2=
\frac{\mean{\tau}^2(\mean{\tau_j}-1)}{w_{j0}^{(\vec{a})}-\mean{\tau_j}}.
\end{align}
or
\begin{align}
w_{j1}^{(\vec{a})}=&\frac{w_{j0}^{(\vec{a})}}{\mean{\tau_j}}
=1+\frac{\mean{\tau}(\mean{\tau}-1)}{\Var(\tau_j)},\nonumber\\
w_{j2}^{(\vec{a})}=&w_{j0}^{(\vec{a})}\frac{\mean{\tau_j}-1}{\mean{\tau_j}}
=(\mean{\tau_j}-1)w_{j1}^{(\vec{a})}.\label{eq:wjka}
\end{align}




\paragraph{Diffusion constants:}
The terms involving $\gamma_j$ in \Eq{Maverage} are
\begin{equation}
  \ln q(\gamma_j)=\text{const.}
  +\ln p(\gamma_j|N)+\sum_{t=1}^{T-1}
  \frac d2\mean{\delta_{j,s_t}}\ln\frac{\gamma_j}{\pi}
  -\mean{\delta_{j,s_t}}(\x_{t+1}-\x_t)^2\gamma_j.
\end{equation}
Hence, if we choose the prior to be a gamma-distribution, then the
variational distribution will be so as well, and we get
\begin{equation}
  q(\gamma_j)=\frac{c_j^{n_j}}{\Gamma(n_j)}\gamma_j^{n_j-1}e^{-c_j\gamma_j},
\end{equation}
with
\begin{equation}\label{VBM_gamma}
  n_j=\tilde n_j+\frac d2\sum_{t=1}^{T-1}\mean{\delta_{j,s_t}},\quad
  c_j=\tilde c_j+\sum_{t=1}^{T-1}\mean{\delta_{j,s_t}}(\x_{t+1}-\x_t)^2,
\end{equation}
and $\tilde n_j$, $\tilde c_j$ being prior parameters. Again, we will
need some averages with respect to $q(\gamma_j)$:
\begin{equation}\label{gammaprop}
  \mean{\gamma_j}_{q(\gamma_j)}=\frac{n_j}{c_j},\quad
  \mean{\ln\gamma_j}_{q(\gamma_j)}=\psi(n_j)-\ln c_j,\quad  
  \Var[\gamma_j]_{q(\gamma_j)}=\frac{n_j}{c_j^2},
\end{equation}
and we can also transform back to the diffusion constant
$D_j=(4\gamma_j\Delta t)^{-1}$, whose distribution is inverse gamma,
\begin{equation}\label{eq:qD}
  q(D_j)=\frac{\beta_j^{n_j}}{\Gamma(n_j)}D_j^{-(n_j+1)}e^{-\beta_j/D_j}.
  \quad \beta_j=c_j/(4\Delta t),
\end{equation}
This means that
\begin{align}
  D^*_{j\;q(D_j)}=&\frac{c}{4(n_j+1)\Delta t},\\
  \mean{D_j}_{q(D_j)}=&\frac{c}{4(n_j-1)\Delta t},\\
  \std[D_j]_{q(D_j)} =&\frac{\mean{D_j}}{\sqrt{n_j-2}}.
\end{align}
The mean value and standard deviation are only defined if $n_j>1,2$
respectively.

The M-step of the iterations, Eq.~\eqref{eq:update1}, thus consists of
updating the variational parameter distributions according to
equations (\ref{VBM_pi}-\ref{VBM_a}) and \eqref{VBM_gamma}. These
equations in turn contain certain averages of the hidden state
distribution $q(s_{1:T-1})$. We now go on and derive the variational
distribution for the hidden state and the E-step, to compute these
averages.
\subsection{Priors}
Directly specifying the parameters of the conjugate priors -- $\tilde
c_j$, $\tilde n_j$, $\tilde w_j^{(\vec{\pi})}$, $\tilde
w_j^{(\vec{a})}$ and $\tilde w_{jk}^{(\matris{B})}$ -- is often an
unintuitive way to express ones prior beliefs. This version of the
code therefore offers different alternatives in terms of, e.g.,
moments of the prior distributions. Different ways to do this can be
chosen via the \texttt{prior\_type\_X} variables in the runinput
files, and more alternatives can be added by
modifying \texttt{VB3\_createPrior.m}. See also
table \ref{Tab:runinput} for details.

\subsection{Hidden states}
Collecting the terms in Eqs. \eqref{xmodel1b} and \eqref{smodel1b} that
depend on the hidden states, we get
\begin{multline}\label{VBE_qs}
 \ln q(s_{1:T-1})=
-\ln Z_s+\mean{\ln p(s_{1:T-1}|\matris{B},\vec{a},\vec{\pi)}}_{q(\matris{A})q(\vec a)q(\vec\pi)}
+\mean{\ln p(\x_{1:T}|s_{1:T-1},\vec{\gamma})}_{q(\vec\gamma)}\\
=-\ln Z_s +\sum_{t=1}^{T-1}\ln H_{t,s_t}+\sum_{t=1}^{T-2}\ln Q_{s_t,s_{t+1}}.
\end{multline}
This looks a like the Hamiltonian of a 1-dimensional spin model in
statistical physics, with external field $\ln H_{t,j}$ and
nearest-neighbor coupling $\ln Q_{jk}$ given by
\begin{align}
  \ln H_{t,j}=& \delta_{1,t}\underbrace{
    \big[\psi(w_j^{(\vec\pi)})-\psi(w_0^{(\vec\pi)})\big]
  }_{\text{initial state distribution}}
  +\frac d2\big(\psi(n_j)-\ln \pi c_j\big)-\frac{n_j}{c_j}(\x_{t+1}-\x_t)^2,
  \\
  \ln Q_{jk}=&\left\{
  \begin{array}{ll}
  \psi(w_{j2}^{(\vec{a})})-\psi(w_{j0}^{(\vec{a})}),&(k=j)\\
  \psi(w_{jk}^{(\matris{B})})-\psi(w_{j0}^{(\matris{B})})
  +
  \psi(w_{j1}^{(\vec{a})})-\psi(w_{j0}^{(\vec{a})})
  ,&(k \ne j)
  \end{array}\right.
\end{align}
Further simplifications in $Q_{jk}$ would obtain if the $\vec a$ and
$\matris{B}$ priors are compatible, since $\tilde
w_{j0}^{(\matris{B})}\equiv \sum_{k\ne j}\tilde
w_{jk}^{(\matris{B})}= \tilde
w_{j1}^{(\vec{a})} \Rightarrow \psi(w_{j0}^{(\matris{B})})
-\psi(w_{j1}^{(\vec{a})})=0$, which would give results equivalent to
the earlier version of vbSPT without the reparameterization in
Eq.~\eqref{eq:aBdef}.

We will need a couple of expectation values to feed back in the next
iteration of the parameter distributions, namely
\begin{equation}
  \mean{\delta_{j,s_t}}_{q(s)}=p(s_t=j),\quad
  \mean{\delta_{j,s_t}\delta_{k,s_{t+1}}}_{q(s)}=p(s_{t+1}=k|s_t=j).
\end{equation}
These can be computed by a dynamic programming
trick\cite{Bishop2006,Mackay1997,Beal2003}, known in the HMM context
as the forward-backward or Baum-Welch
algorithm\cite{Rabiner1989,Baum1972}. As a side effect, one also
obtains the normalization constant $Z_s$, which will be needed below
when computing the lower bound $F$.

\paragraph{Multiple trajectories:}
To analyze many trajectories, one tries to optimize the sum of the
lower bounds for each trajectory with a single model. For the
EM-iterations, this means that the parameters in the variational
distributions get contributions from each trajectory that are just
summed up, \textit{i.e.} averages over the hidden state distribution
gets extended with a summation over $M$ trajectories as
well, \textit{e.g.} in Eq.~\eqref{VBM_B},
\begin{equation}
w_{jk}^{(\matris{B})}=\tilde w_{jk}^{(\matris{B})}
  +\sum_{m=1}^M\sum_{t=1}^{T_m-2}\mean{\delta_{j,s^m_t}\delta_{k,s^m_{t+1}}}_{q(s)},
\end{equation}
where $T_m$ is the number of positions in trajectory $m$. We get a
variational distribution over the hidden states in each trajectory,
where the coupling constants $Q_{jk}$ and external fields $H_{tj}$ are
given by the same parameter distributions (but differ in the
contributions from data).


\subsection{Lower bound}
The lower bound $F$ can be computed just after the
E-step. Substituting our variational ansatz
$q(\theta,s)=q(\theta)q(s)$ into our general expression for $F$,
Eq.~\eqref{eq:F1}, we can exploit the fact that $q(s)$ and $q(\theta)$
are normalized probability distributions, and rewrite it in the form
\begin{multline}
F[q(\theta)q(s),x]=\int d\theta\sum_s \Big[ q(\theta)q(s) \ln
    p(x,s|\theta,N)p(\theta|N) -q(\theta)q(s) \ln q(\theta)q(s) \Big]\\
=\sum_sq(s)\Big[\mean{\ln p(x,s|\theta,N)}_{q(\theta)} -\ln q(s)\Big]
+\int d\theta q(\theta)\Big[\ln p(\theta|N)-\ln q(\theta)\Big].
\end{multline}
Just after the E-step however, $q(s)$ is given by
Eq.~\eqref{eq:update2}, which substituted in the above expressions
leads to cancellations, leaving us with
\begin{equation}
F[q(\theta)q(s),x]=\ln Z_s-\int d\theta q(\theta)\ln\frac{q(\theta)}{p(\theta|N)}.
\end{equation}
The first term is just the normalization constant in
Eq.~\eqref{eq:update2}, which comes out as a by-product of the
forward-backward algorithm. The second term is the negative
Kullback-Leibler divergence of the variational parameter distribution
with respect to the prior. In our case, this distribution factorizes
as in Eq.~\eqref{pfactorization}, and we get a sum of separate and
tractable contributions,
\begin{multline}
\int d\theta q(\theta)\ln\frac{q(\theta)}{p(\theta|N)}
     =\int d\vec\pi q(\vec\pi)\ln\frac{q(\vec\pi)}{p(\vec\pi|N)}
     +\sum_j\int da_j q(a_j)\ln\frac{q(a_j)}{p(a_j|N)}\\
     +\sum_j\int dB_{j,:} q(B_{j,:})\ln\frac{q(B_{j,:})}{p(B_{j,:}|N)}
     +\sum_j\int d\gamma_j q(\gamma_j)\ln\frac{q(\gamma_j)}{p(\gamma_j|N)}.
\end{multline}
For the initial state distribution, we get
\begin{multline}
  \int d\vec{\pi} q(\vec{\pi}) \ln\frac{q(\vec{\pi})}{p(\vec{\pi}|N)}
  =\mean{\ln\left(\frac{\Gamma(w_0^{(\vec{\pi})})}{\Gamma(\tilde w_0^{(\vec{\pi})})}
    \prod_{j=1}^N\frac{\Gamma(\tilde w_j^{(\vec{\pi})})}{\Gamma(w_j^{(\vec{\pi})})}
    \pi_j^{w_j^{(\vec{\pi})}-\tilde w_j^{(\vec{\pi})}}\right)}_{q(\vec{\pi})}\\
%  =\ln\frac{\Gamma(w_0^{(\vec{\pi})})}{\Gamma(\tilde w_0^{(\vec{\pi})})}
%  +\sum_{j=1}^N\left[\ln\frac{\Gamma(\tilde w_j^{(\vec{\pi})})}{\Gamma(w_j^{(\vec{\pi})})}
%    +(w_j^{(\vec{\pi})}-\tilde w_j^{(\vec{\pi})})
%    \big(\psi(w_j^{(\vec{\pi})})-\psi(w_0^{(\vec{\pi})})\big)\right]\\
  =\ln\frac{\Gamma(w_0^{(\vec{\pi})})}{\Gamma(\tilde w_0^{(\vec{\pi})})}
  -(w_0^{(\vec\pi)}-\tilde w_0^{(\vec\pi)})\psi(w_0^{(\vec{\pi})})
  -\sum_{j=1}^N\left[\ln\frac{\Gamma(w_j^{(\vec{\pi})})}{\Gamma(\tilde w_j^{(\vec{\pi})})}
    -(w_j^{(\vec{\pi})}-\tilde w_j^{(\vec{\pi})})\psi(w_j^{(\vec{\pi})})\right],
\end{multline}
where we used Eq.~\eqref{lnpiaverage} for
$\mean{\ln \pi_i}_{q(\vec{\pi})}$. Similarly, the dwell time variable
$a_j$ contributes
\begin{multline}
  \int d\vec{ a} q(\vec{ a}) \ln\frac{q(\vec{ a})}{p(\vec{ a}|N)}
  =\sum_{j=1}^N
  \mean{\ln\left(\frac{\Gamma(w_{j0}^{(\vec{ a})})}{\Gamma(\tilde w_{j0}^{(\vec{ a})})}
    \prod_{k=1}^2\frac{\Gamma(\tilde w_{jk}^{(\vec{ a})})}{\Gamma(w_{jk}^{(\vec{ a})})}
     a_j^{w_{jk}^{(\vec{ a})}-\tilde w_{jk}^{(\vec{ a})}}\right)}_{q(\vec{ a})}\\
  =\sum_{j=1}^N\left\{
  \ln\frac{\Gamma(w_{j0}^{(\vec{ a})})}{\Gamma(\tilde w_{j0}^{(\vec{ a})})}
  -(w_{j0}^{(\vec a)}-\tilde w_{j0}^{(\vec a)})\psi(w_{j0}^{(\vec{ a})})
  -\sum_{k=1}^2\left[\ln\frac{\Gamma(w_{jk}^{(\vec{ a})})}{\Gamma(\tilde w_{jk}^{(\vec{ a})})}
    -(w_{jk}^{(\vec{ a})}-\tilde w_{jk}^{(\vec{ a})})\psi(w_{jk}^{(\vec{ a})})\right]\right\}.
\end{multline}
Note that for an $N=1$ state model, $a_1$ is not well-defined variable
(or $a_1=0$, as there is no other state to escape to), and
$w_{11}^{(\vec{ a})})=0$ which leads to undefined terms in the above
expression. The easiest way out of this is to recognize that the $N=1$
state model has not hidden state dynamics, and hence the contributions
for both $\vec{a}$ and $\matris{B}$ should be omitted (the
contribution from $\vec{\pi}$ does indeed give zero in the $N=1$
case).

The jump matrix rows are also Dirichlet distributed, and each row
contributes
\begin{multline}
  \int dB_{j,:} q(B_{j,:}) \ln\frac{q(B_{j,:})}{p(B_{j,:}|N)}
  =\mean{\ln\left(\frac{\Gamma(w_{j0}^{(\matris{B})})}{\Gamma(\tilde w_{j0}^{(\matris{B})})}
    \prod_{k=1,k\ne j}^N\frac{\Gamma(\tilde w_{jk}^{(\matris{B})})}{\Gamma(w_{jk}^{(\matris{B})})}
    B_{jk}^{w_{jk}^{(\matris{B})}-\tilde w_{jk}^{(\matris{B})}}\right)}_{q(\matris{B})}\\
%  =\ln\frac{\Gamma(w_{j0}^{(\matris{B})})}{\Gamma(\tilde w_{j0}^{(\matris{B})})}
%  +\sum_{k=1,k\ne j}^N\left[\ln\frac{\Gamma(\tilde w_{jk}^{(\matris{B})})}{\Gamma(w_{jk}^{(\matris{B})})}
%    +(w_{jk}^{(\matris{B})}-\tilde w_{jk}^{(\matris{B})})
%    \big(\psi(w_{jk}^{(\matris{B})})-\psi(w_{j0}^{(\matris{B})})\big)
%\right]\\
  =\ln\frac{\Gamma(w_{j0}^{(\matris{B})})}{\Gamma(\tilde w_{j0}^{(\matris{B})})}
  -(w_{j0}^{(\matris{B})}-\tilde w_{j0}^{(\matris{B})})\psi(w_{j0}^{(\matris{B})})\\
  -\sum_{k=1,k\ne j}^N\left[\ln\frac{\Gamma(w_{jk}^{(\matris{B})})}{\Gamma(\tilde w_{jk}^{(\matris{B})})}
    -(w_{jk}^{(\matris{B})}-\tilde w_{jk}^{(\matris{B})})\psi(w_{jk}^{(\matris{B})})
\right].
\end{multline}

Finally, the terms for the precision parameter $\gamma_j$ can be
written in the form
\begin{multline}
\int d\gamma_j q(\gamma_j)\ln\frac{q(\gamma_j)}{p(\gamma_j|N)}
=\mean{\ln\frac{c_j^{n_j}\gamma_j^{n_j-1}e^{-c_j\gamma_j}\Gamma(\tilde n_j)}
      {\tilde c_j^{\tilde n_j}\gamma_j^{\tilde n_j-1}e^{-\tilde c_j\gamma_j}\Gamma(n_j)}}_{q(\gamma_j)}\\
=n_j\ln\frac{c_j}{\tilde c_j}                    
-\ln\frac{\Gamma(n_j)}{\Gamma(\tilde n_j)}
+(n_j-\tilde n_j)\psi(n_j)
-n_j\Big(1-\frac{\tilde c_j}{c_j}\Big),
\end{multline}
where in the last step, we substituted Eq.~\eqref{gammaprop} for
$\mean{\gamma_j}_{q(\gamma_j)}$ and
$\mean{\ln\gamma_j}_{q(\gamma_j)}$.

{\bf ML: still not quite satisfied with the $N=1$ limit of
$a_j$. There should be some way to really have $a_1=0$ for $N=1$.}

\subsection{Matlab notation}
\label{Sec:matlabNotation}
For future reference, we end by listing a translation table between
the notation used in this derivation and the variable names used in
the VB3 code, with the matlab model object named \texttt{W}.

\begin{itemize}
\item The parameters of the parameter variational distributions are
  collected in the \texttt{W.M} and \texttt{W.PM} fields. These are the only fields needed
  to start iterating, all the rest are computed by the algorithm. If
  the \texttt{W.E} field is present, then the \texttt{W.M} field is overwritten in the first
  iteration (and hence the \texttt{W.E} field must be deleted if one wants to use
  the \texttt{W.M} field to parametrize an initial guess).
\begin{center}\begin{tabular}{r|l|r|l|c}
  \textbf{Matlab VB3}& \textbf{This note} & \textbf{Matlab VB3} & \textbf{This note} & \textbf{Eq.} \\
  \hline
\ST  \verb+W.M.wPi(i)+ & $w^{(\vec\pi)}_i$        &
  \verb+W.PM.wPi(i)+& $\tilde w^{(\vec\pi)}_i$ & \eqref{VBM_pi} \\
  \hline
\ST  \verb+W.M.wA(j,k)+ & $w^{(\matris{A})}_{jk}$        &
  \verb+W.PM.wA(j,k)+& $\tilde w^{(\matris{A})}_{jk}$ & \eqref{VBM_A}\\
  \hline
  \verb+W.M.n(j)+ & $n_j$ & 
  \verb+W.PM.n(j)+& $\tilde n_j$ & \eqref{VBM_gamma}\\
  \verb+W.M.c(j)+ & $c_j$ & \verb+W.PM.c(j)+ & $\tilde c_j$ & \\
  \hline
\end{tabular}\end{center}

\item Expectation values computed in the E-step are in the \texttt{W.E}
  fields. Each trajectory gets its own field, and for trajectory \texttt{m},
  the notation is
\begin{center}\begin{tabular}{r|l|c}
\textbf{Matlab VB3}& \textbf{This note} & \textbf{Eq.}\\
\hline
\ST  \verb+W.E(m).wPi(j)+ & $\mean{\delta_{j,s_1}}$ & \eqref{VBM_pi} \\
  \hline
\ST  \verb+W.E(m).wA(j,k)+& $\sum_{t=1}^{T-2}\mean{\delta_{j,s_t}\delta_{k,s_{t+1}}}$ & \eqref{VBM_A}\\
  \hline
\ST  \verb+W.E(m).n(j) +  & $\frac d2\sum_{t=1}^{T-1}\mean{\delta_{j,s_t}}$ & \eqref{VBM_gamma}\\
  \hline
\ST  \verb+W.E(m).c(j) +  & $\sum_{t=1}^{T-1}\mean{\delta_{j,s_t}}(\x_{t+1}-\x_t)^2$ & \eqref{VBM_gamma}\\
  \hline
\end{tabular}\end{center}
\item Misc fields:
\begin{center}\begin{tabular}{l|l}
\hline
\verb+W.N+ & Number of hidden states $N$.\\
\hline
\verb+W.F+ & Total lower bound $F$.\\
\hline
\verb+W.Fterms+ & Various contributions to $F$ (for debugging). \\
\hline
\verb+W.T(j)+ & Length of trajectory j. Note that this counts \\
              &the number of steps, and hence trajectory j will\\
              & contain \verb+W.T(j)++1 particle positions.\\
\hline
\end{tabular}\end{center}
\item Interesting estimates are supplied to describe the output and
  make sense of the converged model and the data. They divided into
  two fields, \texttt{W.est} for quantities that are small and cheap to
  compute (and hence are given every time), while \texttt{W.est2}
  contain quantities that are either large or expensive to compute,
  and therefore only supplied when the iterator algorithm is called
  with the \texttt{'estimate'} argument. When given, the index \texttt{m}
  refers to trajectory number.
  
  \begin{center}\begin{tabular}{l|l|c}
      \textbf{Matlab VB3}& \textbf{This note} & \textbf{Eq.}\\
      \hline
\ST      \verb+W.est.lnQ(j,k)+ & $\ln Q_{j,k}$ & \eqref{VBE_qs}\\
      \hline
\ST      \verb+W.est.Q(j,k)+   & $Q_{j,k}/\max_{j,k}(Q_{j,k})$ & \\
      \hline
\ST       \verb+W.est.Ts(j)+& $\sum_t\mean{\delta_{j,s_t}}$ & \\
      \hline
\ST       \verb+W.est.Ps+& \verb+W.est.Ts/sum(W.est.Ts)+ & \\
      \hline
\ST       \verb+W.est.Amean(j,k)+& $\mean{A_{jk}}_{q(\matris(A)}$ & \eqref{Dirmean} \\
      \hline
\ST       \verb+W.est.Amode(j,k)+& $A^*_{jk\;q(\matris{A})}$& \eqref{Dirmode} \\
      \hline
\ST       \verb+W.est.Astd(j,k)+& $\std[A_{jk\;q(\matris{A})}]$ & \eqref{Dirvar} \\
      \hline
\ST       \verb+W.est.lnAmean(k,j)+&$e^{\verb+W.est.lnAmean+}=\mean{\matris{A}}$ & approx. rate matrix [$\Delta t^{-1}$]\\
      \hline
\ST       \verb+W.est.lnAmode(k,j)+&$e^{\verb+W.est.lnAmode+}=\matris{A}^*$ &approx. rate matrix [$\Delta t^{-1}$]\\
      \hline
\ST       \verb+W.est.dwellMean(j)+& $(1-\mean{A_{jj}})^{-1} $ & dwell time \\
      \hline
\ST       \verb+W.est.dwellMode(j)+& $(1-A^*_{jj})^{-1} $ & dwell time \\
      \hline
\ST       \verb+W.est.gMean(j)+& $\mean{\gamma_j}_{q(\vec\gamma)}$ & \eqref{gammaprop} \\
      \hline
\end{tabular}\end{center}
\begin{center}\begin{tabular}{l|l|c}
    \textbf{Matlab VB3}& \textbf{This note} & \textbf{Eq.}\\
    \hline
    \verb+W.est2.lnH{m}(t,j)+&$\ln H_{t,j} $ & \eqref{VBE_qs} \\
    \hline
    \verb+W.est2.H{m}(t,j)+& $H_{t,j}/ \max_k(H_{t,k})$ & \\
    \hline      
    \verb+W.est2.viterbi{m}+& $ $ & Viterbi path, trj. $m$. \\
    \hline
    \verb+W.est2.sMaxP{m}(j)+& $\mathrm{argmax}_{s_t^m}\mean{\delta_{j,s_t^m}} $ & Most likely states.  \\
    \hline
    \verb+W.est.pst{m}(t,j)+& $\mean{\delta_{j,s_t^m}}$ & Occupation probability $p(s_t^m=j)$.\\
\hline
   
\end{tabular}\end{center}
\end{itemize}




