In this chapter, we describe the statistical method used to analyze
the data, with more details than the corresponding part of the
supplementary information of the original paper \cite{Persson2013}.

%A detailed derivation, including translation tables to interpret the
%analysis code, \FPR{is given in the vbSPT %software
%documentation \cite{vbSPTdoc}.}

Our approach is to model the state kinetics by a hidden Markov Model
(HMM) for diffusing particles with memory-less jumps in diffusion
constants, which we analyze by an approximate approach to a maximum
evidence model selection, known as variational Bayes or ensemble
learning \cite{Mackay2003,Bishop2006}. Variational algorithms for HMMs
have been derived
earlier \cite{Mackay1997,Ghahramani2002,Beal2003,Bronson2009,Bronson2010},
and applied successfully in a biophysical setting to \textit{e.g.}
single molecule FRET data \cite{Bronson2009,Bronson2010}. The main
advantage of the variational maximum evidence approach over the more
common maximum likelihood approach to HMMs is the inherent complexity
control, \textit{i.e.} the ability to not only learn parameter values
from data, but also to make model selection and learn the number of
hidden states \cite{Bronson2010}.


\subsection{Model selection by maximum evidence}
In this section, we briefly introduce our model selection criteria,
maximum evidence.  For simplicity, we use $x$, $s$, $\theta$, and $N$
to denote tracking data, hidden states corresponding to the data,
unknown parameters in the model and the number of hidden states,
respectively. For readers new to Bayesian statistics, we also
recommend the brief introduction by Eddy \cite{Eddy2004} (or
textbooks\cite{Mackay2003,Bishop2006}).

A probabilistic model specifies the probability of obtaining the data
$x$ and hidden states $s$, given a model $N$ and some parameter values
$\theta$: $p(x,s|\theta,N)$. Our particular model will be specified in
Sec.~\ref{Sec:esimodel} below.  To use this for Bayesian model
selection, we treat all variables in this function as random
variables, use the laws of probability to derive the inverse
probability $p(N|x)$, which we interpret as a statement about our
degree of confidence in model $N$ given the data $x$, and prefer the
most likely model.

This will require us to specify prior distributions, of the form
$p(\theta,N)$, that express our beliefs about the models and their
parameters prior to seeing the data $x$.  Indeed, the willingness to
assign a probability distribution to unknown parameters and treat them
on an equal footing with other random variables is a characteristic of
Bayesian statistics.

Returning to the derivation of $p(N|x)$, we start by computing the
weighted average over all possible hidden states, known as marginalizing
over them (or, in physics jargon, integrating or summing them out):
\begin{equation}\label{eq:s_marginalize}
  p(x|\theta,N)=\sum_s p(x,s|\theta,N).
\end{equation}
Using Bayes' rule on $p(x|\theta,N)$, and introducing the
aforementioned prior distributions, we can derive the joint model and
parameter probability as
\begin{equation}\label{eq:posterior}
p(\theta,N|x)=\frac{p(x|\theta,N)p(\theta,N)}{p(x)},
\end{equation}
where the denominator is a normalization constant,
\begin{equation}
p(x)=\sum_{N'}\int d\theta' p(x|\theta',N')p(\theta|N')p(N').
\end{equation}
Finally, the sought inverse probability is 
obtained by marginalizing over the parameters,
\begin{equation}\label{eq:pnx}
p(N|x)=\frac{\int d\theta p(x|\theta,N)p(\theta,N)}{p(x)}
=\frac{1}{p(x)}\int d\theta\sum_s p(x,s|\theta,N)p(\theta|N)p(N),
%}{\sum_{N'}\int d\theta'
%p(x|\theta',N')p(\theta|N')p(N')},
\end{equation}
where we also re-wrote the joint probability in the form
$p(\theta,N)=p(\theta|N)p(N)$.

If we further assume that $p(N)$ is constant, \textit{i.e.} that all models
(in some interval $0<N\le N_\text{max}$) are equally probable \emph{a
priori}, and note that the denominator $p(x)$ is independent of
parameters and models, then the best model is the one that maximizes the
numerator in \Eq{eq:pnx}, also known as the \emph{evidence},
\begin{equation}\label{eq:MEdef}
N_\text{ME}=
\mathrm{argmax}_N \int d\theta\sum_s p(x,s|\theta,N)p(\theta|N),
\end{equation}
where argmax$_y$ $f(y)$ denotes the value of $y$ that maximizes the
function $f$. 

%Note that this can be seen as a generalization of a maximum likelihood
%estimator of a parameter value for a given model, which with analogous
%notation and assumed a constant prior density $p(\theta)$ can be
%written
%\begin{equation}
%\theta_\text{ML}=\mathrm{argmax}_\theta \sum_s p(x,s|\theta).
%\end{equation}

%\FPR{ML: Dropped any reference to maximum likelihood. OK?}

\subsection{Variational maximum evidence}
In practice, the integrals and sums in the evidence are intractable
for almost all interesting models, and further progress requires good
approximations and computers. The difficulties resemble those of
computing partition functions in statistical physics, and the two
common fallbacks in that field -- Monte Carlo
simulations \cite{Green1995,Robert2000,Mackay2003} and mean field
theory \cite{Mackay1997,Mackay2003,Bishop2006} -- work here as well.
The variational or ensemble learning approach we will follow corresponds to
mean field theory, where we seek approximations to the
posterior distribution in \Eq{eq:posterior} in the form
\begin{equation}
p(s,\theta|x,N)=\frac{p(s,x|\theta,N)p(\theta|N)}{p(x|N)}
\approx q(s)q(\theta),
\end{equation}
\textit{i.e.}, where the hidden states $s$ and parameter values $\theta$ are
statistically independent.  A general recipe to derive mean-field-like
approximations of this type\cite{Mackay2003} is to rewrite the
logarithm of the evidence as the solution of an optimization problem
in an arbitrary distribution $q(\theta,s)$ over the unknowns. The
approximation then consists of optimizing in a restricted space, in
this case that of separable distributions of the form $q(\theta)q(s)$.

To begin with, we multiply and divide the evidence by $q(\theta,s)$,
and then make use of Jensen's inequality to write
\begin{multline}\label{eq:F1}
\ln p(x|N)=\ln \int d\theta\sum_s 
    q(\theta,s)\frac{p(x,s|\theta,N)p(\theta|N)}{q(\theta,s)}\\
    \ge \int d\theta\sum_s q(\theta,s) 
    \ln \frac{p(x,s|\theta,N)p(\theta|N)}{q(\theta,s)}\\
    =\int d\theta\sum_s \Big[
    q(\theta,s) \ln p(x,s|\theta,N)p(\theta|N)
    -q(\theta,s) \ln q(\theta,s)
    \Big]\equiv F[q(\theta,s),x].
\end{multline}
This inequality is true for any distribution $q(\theta,s)$. If we use
the calculus of variations to directly optimize $F$ with respect to
$q(\theta,s)$ subject to a normalization constraint (since
$q(\theta,s)$ is a probability distribution), we find
$q^\dagger(\theta,s)=\frac{p(x,s|\theta,N)p(\theta|N)}{p(x|N)}$. Substituting
back in Eq.~\eqref{eq:F1}, we then get $F[q^\dagger,x]=\ln
p(x|N)$, \textit{i.e.} the original intractable problem.

Moreover, we can rewrite the second line of \Eq{eq:F1} as the evidence
minus a Kullback-Leibler divergence,
\begin{multline}
F[q(\theta,s),x]=\ln p(x|N)
-\int d\theta\sum_s q(\theta,s) 
\ln \frac{q(\theta,s)}{\frac{1}{p(x|N)}p(x,s|\theta,N)p(\theta|N)}\\
=\ln p(x|N)
-D_{KL}\big(q(\theta,s)||\frac{p(x,s|\theta,N)p(\theta|N)}{p(x|N)}\big).
\end{multline}
Hence, maximizing $F$ means minimizing the KL-divergence from $q^\dagger$ to
$q$, and it follows from the properties of KL-divergencies\cite{wiki:KLdiv}
that $q^\dagger$ is the only distribution that achieves the best bound
$F=\ln p(x|N)$. 

To make this a useful approximation, we place restrictions on the
variational distribution $q(\theta,s)$. Our approximate model
selection will be to prefer the model for which the tightest bound $F$
can be found within that restricted function space. KL-divergencies
are not proper metrics, but have enough metric-like properties to
motivate using the corresponding optimal distribution $q^*(\theta,s)$
for approximate inference about the parameter values and hidden
states, not only model selection\cite{Mackay2003, Bishop2006}.

A useful restriction is that of separable distribution
$q(\theta,s)\approx
q(\theta)q(s)$ \cite{Mackay1997,Beal2003,Bronson2009}, which does give
a tractable problem. In particular, optimizing with respect to
$q(\theta)$ and $q(s)$, while using Lagrange multipliers to enforce
normalization, one can derive the following equations:
\begin{align}
  \ln q(\theta)=&-\ln Z_\theta+\ln p(\theta|N)
  +\mean{\ln p(x,s|\theta,N)}_{q(s)},\label{eq:update1}\\
  \ln q(s)=&-\ln Z_s+\mean{\ln p(x,s|\theta,N)}_{q(\theta)},\label{eq:update2}
\end{align}
where $Z_\theta$ and $Z_s$ are normalization constants, and
$\mean{\cdot}_{q(\cdot)}$ denotes an average with respect to
$q(\cdot)$. As it turns out, these averages can be computed for our
model. Following earlier variational treatments of
HMMs \cite{Mackay1997,Beal2003,Bronson2009}, we solve these equations
iteratively, by alternatingly updating $q(s)$ and $q(\theta)$ for fixed
$q(\theta)$ and $q(s)$, respectively. The resulting algorithm greatly resembles the
well-known expectation-maximization procedure for maximum-likelihood
optimization in HMMs \cite{Ghahramani2002,Rabiner1989,Baum1972} (see
Sec.~\ref{Sec:derivation} for details).

\subsection{Diffusion model}
\label{Sec:esimodel}
We assume that we can track the particles in $d$ dimensions, and call
$\x_t$ the position at time $t$.  The time between consecutive
measurements is $\Delta t$, but we are going to use $t$ as an integer
index as well. We model the position by
simple diffusion,
\begin{equation}\label{eq:diff1}
        \x_{t+1}=\x_t+\sqrt{2D_{s_t}\Delta t}\vec{w}_t,
\end{equation}
where $\vec{w}_t$ are independent $d$-dimensional Gaussian variables
with uncorrelated components of zero mean and unit variance.

Binding and unbinding events are modeled as jumps in the diffusion
constant $D_{s_t}$, as indicated by the degree of freedom
\mbox{$s_t\in\{1,2,\ldots N\}$}, where $N$ is the number of diffusive 
states. This degree of freedom constitutes our hidden state, and we
model it as a discrete Markov process with a transition matrix
$\matris{A}$, and initial state probabilities
$\vec{\pi}$, \textit{i.e.}
\begin{equation}\label{eq:Aij}
p(s_1=j)=\pi_j,\quad 
p(s_t=j|s_{t-1}=i)=A_{ij}\text{, for $t>1$},
\end{equation}
where normalization demands
$\sum_{k=1}^N\pi_k=\sum_{k=1}^NA_{jk}=1$. The number of hidden states
$N$ is a parameter to be determined from the data.  The vector of
diffusion constants corresponding to the different hidden states is
denoted $\vec D=(D_1,D_2,\ldots,D_N)$. We will also use the
shorthand notation $\x_{1:T}=\{\x_1,\x_2,\ldots , \x_T\}$ and
$s_{1:T-1}=\{s_1,s_2,\ldots,s_{T-1}\}$ for the positions and hidden
states of a whole trajectory.\footnote{Since $s_T$ does not influence
the position data, we exclude it from the model.}

For our analysis, we need expressions for the joint distribution of
positions, hidden states, and parameters,
$p(\x_{1:T},s_{1:T-1},\vec{D},\matris{A},\vec\pi|N)$, which can be
factorized as

\begin{multline}\label{eq:efactor1}
p(\x_{1:T},s_{1:T-1},\vec{D},\matris{A},\vec\pi|N)
=p(\x_{1:T}|s_{1:T-1},\vec{D})
p(s_{1:T-1}|\matris{A},\vec\pi)\\
\times p(\vec{D}|N)
p(\matris{A}|N)
p(\vec\pi|N).
%p(\vec{D},\matris{A},\vec\pi|N).
\end{multline}

The first two factors on the above right hand side are specified by
the model. In particular, the distribution of hidden state sequences
is
\begin{equation}\label{smodel1}
  p(s_{1:T-1}|\matris{A},\vec{\pi})=\pi_{s_1}\prod_{t=1}^{T-2}A_{s_ts_{t+1}}
  =\prod_{m=1}^N\pi_m^{\delta_{m,s_1}}
  \prod_{t=1}^{T-2}\prod_{k,j=1}^N A_{kj}^{\delta_{k,s_t}\delta_{j,s_{t+1}}},
\end{equation}
and the distribution of positions, conditional on the hidden states, is
given by 
\begin{multline}
  p(\x_{1:T}|s_{1:T-1},\vec
  D)=p(\x_1)\prod_{t=1}^{T-1}p(\x_{t+1}-\x_t|s_t,\vec D)\\
  =p(\x_1)\prod_{t=1}^{T-1} \frac{1}{(4\pi D_{s_t}\Delta
  t)^{d/2}}e^{-\frac{1}{4D_{s_t}\Delta t}\Delta\x_t^2}.
\end{multline}
In these equations, $\delta_{j,k}$ is the Kronecker delta. The initial
position distribution $p(\x_1)$ does not depend on any parameter of
interest to us here, and we will drop it from the analysis.  It is
also convenient to introduce the inverse diffusion constant
$\gamma_j=1/4D_j\Delta t$, and write
\begin{align}
    p(\x_{1:T}|s_{1:T-1},\vec\gamma)=&
    \prod_{t=1}^{T-1}\left(\frac{\gamma_{s_t}}{\pi}\right)^{d/2}
    e^{-\gamma_{s_t}(\x_{t+1}-\x_t)^2}\nonumber\\
    =&\prod_{t=1}^{T-1}\prod_{k=1}^N\left(
    \frac{\gamma_k}{\pi}\right)^{\frac d2\delta_{k,s_t}}
    e^{-\delta_{k,s_t}\gamma_k(\x_{t+1}-\x_t)^2}. 
    \label{xmodel1}
\end{align}

The last three factors in Eq.~\eqref{eq:efactor1} are the prior
distributions for the parameters, expressing our beliefs about the
parameter values for different model sizes (values of $N$), before
seeing the data. For computational convenience, independent
priors, \textit{e.g.}  $p(\vec{\gamma}|N) p(\matris{A}|N)
p(\vec\pi|N)$, are used instead of the more general
$p(\vec{\gamma},\matris{A},\vec\pi|N)$. For the same reason priors
functional forms are chosen as conjugate priors (see Chap.2.4.2 in
Bishop\cite{Bishop2006}). As we will see in the derivation below, this
means gamma distributions for the inverse diffusion constant priors,
and Dirichlet distributions (a multidimensional version of the beta
distribution) for the initial state and transition probability priors.
Within these constraints, we strive to choose uninformative, or weak,
priors in order to let the data speak for itself as much as possible.

\subsection{Treatment of many trajectories}
Typical \textit{in vivo} single particle tracking experiments produce
many rather short trajectories. Most trajectories contain less than 20
consecutive positions, and probably very few ($\le 2$) transitions. If
considered in isolation, such trajectories are not very informative,
and transitions are difficult to identify accurately. To extract
meaningful information, we need to pool many trajectories. The
simplest way to do that is to assume that they are statistically
independent and governed by the same model and parameter
set, \textit{i.e.}  that all molecules (including those from
different cells in the same batch) are equivalent and that the
interaction dynamics do not change over time. In that case, the
probability distribution for a set of $M$ trajectories is a product of
single-trajectory densities governed by the same model,
\begin{equation}\label{eq:pxs}
p(\{\x^i_{1:T},s^i_{1:T-1}\}_{i=1}^M|\vec{\gamma},\matris{A},\vec\pi,N)
=\prod_{i=1}^Mp(\x^i_{1:T},s^i_{1:T-1}|\vec{\gamma},\matris{A},\vec\pi,N),
\end{equation}
where different trajectories are indicated by the index $i$.

%The independence assupmtion is easy to justify, since each trajectory
%corresponds to a new flourophore.  The homogeneity assumption is less
%obvious, and will be explored in further depth among the controls
%below\footnote{ML: so far, I am hopeful that we will get around to
%actually doing it, and this footnote is a reminder.}.


%Although a detailed derivation of the variational algorithm is
%deferred until in {\bf X}, we nevertheless quote the variational
%parameter distributions here, as well as some posterior expectation
%values we use for to describe the converged model.

%The initial state probability $\vec\pi$ is Dirichlet
%distributed \cite{wiki:dirichlet} with parameters $\vec
%w^{(\pi)}=(w_1^{(\pi)},w_2^{(\pi)},\ldots,w_N^{(\pi)})$, i.e.,
%\begin{equation}
%\end{equation}
