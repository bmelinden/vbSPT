%\LTcapwidth=\textwidth


\begin{longtable}{|m{0.25\textwidth}|>{\centering\arraybackslash}m{0.3\textwidth}|m{0.45\textwidth}|}
\captionsetup{margin=-0.6cm}
\caption{Explanations of the parameters defined in the runinputfiles. Num stands for numerical value.}\label{Tab:runinput} \\
\hline \multicolumn{1}{|>{\centering\arraybackslash}m{0.25\textwidth}|}{\textbf{Parameter}} & 
\multicolumn{1}{>{\centering\arraybackslash}m{0.3\textwidth}|}{\textbf{Value}} & 
\multicolumn{1}{>{\centering\arraybackslash}m{0.45\textwidth}|}{\textbf{Description}} \\ \hline\hline
\endfirsthead

\multicolumn{3}{m{9cm}}%
{{ \tablename\ \thetable{} -- continued from previous page}} \\
\hline \multicolumn{1}{|>{\centering\arraybackslash}m{0.25\textwidth}|}{\textbf{Parameter}} & 
\multicolumn{1}{>{\centering\arraybackslash}m{0.3\textwidth}|}{\textbf{Value}} & 
\multicolumn{1}{>{\centering\arraybackslash}m{0.45\textwidth}|}{\textbf{Description}} \\ \hline\hline
\endhead

\multicolumn{3}{|r|}{{Continued on next page}} \\ \hline
\endfoot

\hline
\endlastfoot


\textbf{inputfile}						&	  'filename'						&		Name of the .mat file that contains the trajectories.\\ [0.5ex]  \hline 
\textbf{trajectoryfield}				&	  'trajfield'							&		Name of the field in the .mat file that contains the trajectories to be analysed.\\ [0.5ex]  \hline 
\textbf{parallelize\_config}		&	  true/false						&		Determines whether parallel computing should be used.\\ [0.5ex]  \hline 
\textbf{parallel\_start}				&	  'command'						&		Command used to start the parallelization.\\ [0.5ex]  \hline 
\textbf{parallel\_end}				&	  'command'						&		Command used to end the parallelization.\\ [0.5ex]  \hline 
\textbf{outputfile	}					&	  'filename'						&		Name of the .mat file where the results are saved.\\ [0.5ex]  \hline 
\textbf{jobID}							&	  'description'					&		Description of the job for your own records.\\ [0.5ex]  \hline 
\textbf{timestep}						&	  Num								&		Timestep between points in the trajectories, given in [s].\\ [0.5ex]  \hline 
\textbf{dim}								&	  Num								&		Dimensionality of the data to be analysed (the first dim columns in the coordinate matrix will be used).\\ [0.5ex]  \hline 
\textbf{trjLmin}						&	  2									&		Minimum length of trajectories to be included in the analysis. Recommended to keep at the default value of 2.\\ [0.5ex]  \hline 
\textbf{runs}							&	  Num								&		Number of analysis attempts at each model size. Recommended to use a multiple of the number of cores when running in parallel.\\ [0.5ex]  \hline 
\textbf{maxHidden}					&	  Num								&		Maximum number of hidden states to consider. Recommended to use twice the amount of expected hidden states.\\ [0.5ex]  \hline 
\textbf{maxIter}						&	  []									&		Maximum number of VB iterations. Recommended to keep as [] which uses the default value of 1000.\\ [0.5ex]  \hline 
\textbf{relTolF}						&	  []									&		Convergence criterion for the relative change in likelihood bound. Recommended to keep as [] which uses the default value of 1e$^{-8}$.\\[0.5ex]  \hline 
\textbf{tolPar}							&	  []									&		Convergence criterion for the M-step parameters. Recommended to keep as [] which uses the default value of 1e$^{-2}$.\\[0.5ex]  \hline 
\textbf{stateEstimate}				&	  true/false						&		Determines whether extra large and computing intensive estimates, including Viterbi paths, should be computed.\\[0.5ex]  \hline 
\textbf{bootstrapNum}			&	  Num/0							&		Number of bootstrap resamplings. Set to 0 to disable bootstrapping.\\ [0.5ex]  \hline 
\textbf{fullBootstrap}				&	  true/false						&		Determines whether bootstrapping should be done for all model sizes, not only the best global model.\\[0.5ex]  \hline 
\textbf{init\_D	}						&	  [Num Num]					&		Interval for initial guess of diffusion coefficients, given in [$L^2s^{-1}$]. $L$ is the length unit for the input data.\\ [0.5ex]  \hline 
\textbf{init\_tD}						&	  [Num Num]					&		Interval for initial guess of dwell times (lifetimes) of the states, given in [$s$]. Recommended to use 2-20 times the timestep.\\ [0.5ex]  \hline 
\textbf{prior\_piStrength}		&	  5									&		Prior strength for the initial state distribution (assumed uniform), given in pseudocounts and should be set low. Recommended to keep the default value of 5.\\ [0.5ex]  \hline 
\textbf{prior\_D}						&	  Num								&		Diffusion coefficient prior value, given in [$L^2s^{-1}$]. $L$ is the length unit for the input data.\\ [0.5ex]  \hline 
\textbf{prior\_Dstrength}			&	  5									&		Strength for the diffusion coefficient prior. Recommended to keep the default value of 5.\\ [0.5ex]  \hline 
\textbf{prior\_tD}						&	  10*timestep					&		Dwell time prior value, given in [$s$].\\ [0.5ex]  \hline 
\textbf{prior\_tDstrength}		&	  2*prior\_tD/timestep		&		Strength for the dwell time prior. Recommended to keep the default value of twice the dwelltime prior in timesteps.\\ [1ex] % [1ex] adds vertical space
\end{longtable}


\newpage

\begin{table}[ht]
\caption{Important variables within the Wbest structure. The states in the model is always sorted after increasing diffusion coefficient.} 
\centering 
\begin{tabular*}{13.1cm}{ | m{2.7cm} | m{9.5cm} | } 
\hline
\texttt{Wbest.} & \textbf{Description} \\ [0.5ex] 
\hline \hline 
\texttt{dim} 						& Dimensionality of the analysed data.\\[0.5ex]   \hline 
\texttt{N}  						& Number of states in the model.\\[0.5ex]  \hline 
\texttt{T}							& The trajectrory length for all trajectories used in the analysis.\\[0.5ex]  \hline 
\texttt{F}						 	& The score of the model.\\[0.5ex]  \hline 
\texttt{est.Ptot}				& The occupation of each state.\\[0.5ex]  \hline 
\texttt{est.Amean}			& The transition matrix. States the probability of going between two states within a timestep, \textit{i.e.} Amean(2, 3) gives the probability that a molecule in state 2 will transition to state 3 within a timestep.\\[0.5ex]  \hline 
\texttt{dwellMean}			& The mean dwelltime/lifetime for each state, given in units of timesteps.\\[0.5ex]  \hline 
\texttt{DdtMean}				& D*timestep in the same length units as the data put into the analysis.\\[0.5ex]  \hline 
\texttt{est2.viterbi}			& The Viterbi path for each trajectory. The est2 field is only computed if 'stateEstimate=true' is given in the runinput file (equivalent of choosing 'Additional estimates' in the GUI) since it is rather large and computer intensive. \\[1ex] % [1ex] adds vertical space
\hline 
\end{tabular*}\label{Tab:Wbest} 
\end{table}

\newpage

\begin{table}[ht]
\caption{Important variables within the bootstrap structure.} 
\centering 
\begin{tabular*}{13.1cm}{ | m{2.7cm} | m{9.5cm} | } 
\hline
\texttt{bootstrap.} 			& \textbf{Description} \\ [0.5ex] 
\hline\hline 
\texttt{wbs} 						& All the individual bootstraps containing score (F), the .est field (see Tab. \ref{Tab:Wbest}) and the index of the randomly chosen trajectories (ind).\\[0.5ex]   \hline 
\texttt{Wmean.est}  						& The mean value from all the bootstraps for the .est field from Tab. \ref{Tab:Wbest}.\\[0.5ex]  \hline 
\texttt{Wstd}				 	& The mean value from all the bootstraps for the .est field from Tab. \ref{Tab:Wbest}.\\[0.5ex]  \hline 
\texttt{WmeanN}				& Same as Wmean but for all model sizes. Computed if 'fullBootstrap=true' was given in the runinput file (or chosen in the GUI).\\[0.5ex]  \hline 
\texttt{WstdN}					& Same as Wstd but for all model sizes. Computed if 'fullBootstrap=true' was given in the runinput file (or chosen in the GUI).\\[0.5ex]  \hline 
\texttt{Fbootstrap}			& The score for each bootstrap andmodel size. Computed if 'fullBootstrap=true' was given in the runinput file (or chosen in the GUI).\\[0.5ex]  \hline 
\texttt{pBest}					& The part of all bootstraps that resulted in the different model sizes, \textit{i.e.} pBest(3)=0.8 means that 80\% of the bootstraps resulted in a 3 state model.\\[1ex] % [1ex] adds vertical space
\hline 
\end{tabular*}\label{Tab:bootstrap} 
\end{table}


\newpage