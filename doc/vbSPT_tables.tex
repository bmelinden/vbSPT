%\LTcapwidth=\textwidth


\begin{longtable}{|m{0.25\textwidth}|>{\centering\arraybackslash}m{0.3\textwidth}|m{0.45\textwidth}|}
%\captionsetup{margin=-0.6cm}
\caption{Explanations of the parameters defined in the
  runinputfiles. Int and real denotes integers and real numbers
  respectively, most of which need to be positive if not state
  otherwise. The units of some quantities are indicated as [time],
  [length$^2$/time], etc, indicating the need to choose units
  consistent with the data.}\label{Tab:runinput} \\ \hline
\multicolumn{1}{|>{\centering\arraybackslash}m{0.25\textwidth}|}{\textbf{Parameter}}
&
\multicolumn{1}{>{\centering\arraybackslash}m{0.3\textwidth}|}{\textbf{Value}}
&
\multicolumn{1}{>{\centering\arraybackslash}m{0.45\textwidth}|}{\textbf{Description}}
\\ \hline\hline \endfirsthead

\multicolumn{3}{m{9cm}}%
{{ \tablename\ \thetable{} -- continued from previous page}} \\
\hline \multicolumn{1}{|>{\centering\arraybackslash}m{0.25\textwidth}|}{\textbf{Parameter}} & 
\multicolumn{1}{>{\centering\arraybackslash}m{0.3\textwidth}|}{\textbf{Value}} & 
\multicolumn{1}{>{\centering\arraybackslash}m{0.45\textwidth}|}{\textbf{Description}} \\ \hline\hline
\endhead

\multicolumn{3}{|r|}{{Continued on next page}} \\ \hline
\endfoot

\hline
\endlastfoot


\texttt{inputfile} & 'filename' & Name of the .mat file that contains the trajectories.\\ 
[0.5ex] \hline 
\texttt{trajectoryfield} & 'trajfield' & Name of the field in the .mat file that contains the
trajectories to be analysed.\\ 
[0.5ex] \hline
\texttt{parallelize\_config} & true/false & Determines whether parallel computing should be used.\\ 
[0.5ex] \hline
\texttt{parallel\_start} & 'command' & Command used to start the parallelization.\\ 
[0.5ex] \hline 
\texttt{parallel\_end} & 'command' & Command used to end the parallelization.\\ 
[0.5ex] \hline 
\texttt{outputfile } & 'filename' & Name of the .mat file where the results are saved.\\ 
[0.5ex] \hline
\texttt{jobID} & 'description' & Description of the job for your own records.\\ 
[0.5ex] \hline \texttt{timestep} & real & Timestep between points in the trajectories, in units of [time].\\ 
[0.5ex] \hline
\texttt{dim} & int & Dimensionality of the data to be analysed (the first dim columns in the coordinate data will be used).\\ 
[0.5ex] \hline 
\texttt{trjLmin} & int & Minimum length of trajectories to be included in the analysis. Recommended default value: 2.\\ 
[0.5ex] \hline 
\texttt{runs} & int & Number of analysis attempts at each model size. Recommended to use a multiple of the number of cores when running in parallel.\\ 
[0.5ex] \hline 
\texttt{maxHidden} & int & Maximum number of hidden states to consider. Recommended to use twice the amount of expected hidden states.\\ 
[0.5ex] \hline
\texttt{maxIter} & int & Maximum number of VB iterations. Set to an empty matrix (\texttt{[]}) to use the recommended default value of
1000.\\ [0.5ex] \hline \texttt{relTolF} & real & Convergence criterion
for the relative change in likelihood lower bound. Recommended to set
to \texttt{[]}, which uses the default value of $10^{-8}$.\\[0.5ex]
\hline \texttt{tolPar} & real & Convergence criterion for the M-step
parameters. Recommended to set to \texttt{[]}, which uses the default
value of $10^{-2}$.\\[0.5ex] \hline \texttt{stateEstimate} &
true/false & Determines whether extra large and computer intensive
estimates, including Viterbi paths, should be computed.\\[0.5ex]
\hline \texttt{bootstrapNum} & int/0 & Number of bootstrap
resamplings. Set to 0 to disable bootstrapping. 100 or more
resamplings are recommended when using this feature.\\ [0.5ex] \hline
\texttt{fullBootstrap} & true/false & Determines whether bootstrapping
should be done for all model sizes, not only the best global
model. Bootstrapping all model sizes requires more computer time, but
can give an indication of how robust the model size estimate
is.\\[0.5ex] \hline \texttt{init\_D } & [real real] & Interval for
initial guess of diffusion coefficients, given in unit of
[length$^2$/time]. \\ [0.5ex] \hline \texttt{init\_tD} & [real real] &
Interval for initial guess of mean dwell times (lifetimes) of the
hidden states, given in units of [time]. Recommended default value:
\mbox{\texttt{[2 20]*timestep}}. Guessing too short dwell times does
not hurt convergence.\\ [0.5ex] 

\hline \hline

\texttt{prior\_type\_D} & string & Type of parameterization for
diffusion constant. Default: 'mean\_strength' (used by
\citet{Persson2013}). \\ [0.5 ex] \hline

'mean\_strength':
\texttt{prior\_D} & real & Diffusion coefficient prior mean value, in
units of [length$^2$/time]. An order of magnitude estimate is good
enough.\\ [0.5ex] \hline 
'mean\_strength':
\texttt{prior\_Dstrength} & int & Strength of the diffusion
coefficient prior. Recommended default value: 5.\\ [0.5ex] 

\hline\hline

\texttt{prior\_type\_Pi} & string & Type of parameterization for
initial state prior. Alternatives: 'flat', and 'natmet13' (default,
used by \citet{Persson2013}). \\ [0.5 ex] \hline

'flat': no parameteres needed. && A flat Dirichlet prior;
$\tilde w_j^{(\vec{\pi})}=1$. \\ [0.5 ex] \hline

'natmet13': \texttt{prior\_piStrength} & int & Prior strength $\tilde
w_0^{(\vec{\pi})}$ for the initial state probability, , given in
pseudocounts, $\tilde w_j^{(\vec{\pi})}=\tilde w_0^{(\vec{\pi})}/N$.
Default value: 5.  \\ [0.5ex]

\hline\hline

\texttt{prior\_type\_A}& string & Type of parameterization for
transition matrix. Alternatives: 'dwell\_Bflat', and 'natmet13'
(default, used by \citet{Persson2013}).\\ [0.5 ex] \hline

'dwell\_Bflat' & & Specifies mean value and standard deviation of the
prior mean dwell times (same for all states) to specify $\tilde
w_j^{(\vec{a})}$, and uses a flat Dirichlet prior for the rows of
$\matris{B}$. \\ [0.5ex] \hline

'dwell\_Bflat': \texttt{prior\_tD} & real & Dwell time
prior mean value, in units [time]. Must not be smaller than
\texttt{2*timestep}. Default: \texttt{10*timestep}. \\ [0.5ex] \hline

'dwell\_Bflat': \texttt{prior\_tDstd} & real& Standard deviation of
prior mean dwell time, in units [time]. Set a high value for an
unformative (weak) prior. Default \texttt{100*timestep}.
\\ [0.5ex] \hline

'natmet13' & & Specifies a Dirichlet prior for all rows of the
transition matrix $\matris{A}$ directly, via the mean dwell time and
total strength $\tilde w_{j0}^{(\matris{A})}=\tilde
w_{j0}^{(\vec{a})}$, while using a uniform prior for $\matris{B}$ with
$\tilde w_{j0}^{(\matris{B})}=\tilde w_{j1}^{(\vec{a})}$.  \\ [0.5ex]
\hline


'natmet13': \texttt{prior\_tD} & real & Dwell time
prior mean value, in units [time]. Must not be smaller than
\texttt{2*timestep}. Default: \texttt{10*timestep}. \\ [0.5ex] \hline

'natmet13': \texttt{prior\_tDstrength} & real & Strength of the transition
probability prior. Recommended default value:
\texttt{2*prior\_tD/timestep}\\ [1ex] % [1ex] adds vertical space
\end{longtable}

\newpage

\begin{table}[ht]
\caption{Important variables within the Wbest structure. The states in the model is always sorted after increasing diffusion coefficient.} 
\centering 
\begin{tabular*}{13.1cm}{ | m{2.7cm} | m{9.5cm} | } 
\hline
\texttt{Wbest.} & \textbf{Description} \\ [0.5ex] 
\hline \hline 
\texttt{dim} 						& Dimensionality of the analysed data.\\[0.5ex]   \hline 
\texttt{N}  						& Number of states in the model.\\[0.5ex]  \hline 
\texttt{T}							& The trajectrory length for all trajectories used in the analysis.\\[0.5ex]  \hline 
\texttt{F}						 	& The score of the model.\\[0.5ex]  \hline 
\texttt{est.Ptot}				& The occupation of each state.\\[0.5ex]  \hline 
\texttt{est.Amean}			& The transition matrix. States the probability of going between two states within a timestep, \textit{i.e.} Amean(2, 3) gives the probability that a molecule in state 2 will transition to state 3 within a timestep.\\[0.5ex]  \hline 
\texttt{dwellMean}			& The mean dwelltime/lifetime for each state, given in units of timesteps.\\[0.5ex]  \hline 
\texttt{DdtMean}				& D*timestep in the same length units as the data put into the analysis.\\[0.5ex]  \hline 
\texttt{est2.sMaxP}			& The sequence of most likely hidden states for each trajectory. The est2 field is only computed if \texttt{stateEstimate=true;} is given in the runinput file (equivalent of choosing 'Additional estimates' in the GUI) since it is rather large and computer intensive. \\[1ex] % [1ex] adds vertical space
\hline 
\end{tabular*}\label{Tab:Wbest} 
\end{table}

\newpage

\begin{table}[ht]
\caption{Important variables within the bootstrap structure.} 
\centering 
\begin{tabular*}{13.1cm}{ | m{2.7cm} | m{9.5cm} | } 
\hline
\texttt{bootstrap.} 			& \textbf{Description} \\ [0.5ex] 
\hline\hline 
\texttt{wbs} 						& All the individual bootstraps containing score (F), the .est field (see Tab. \ref{Tab:Wbest}) and the index of the randomly chosen trajectories (ind).\\[0.5ex]   \hline 
\texttt{Wmean.est}  						& The mean value from all the bootstraps for the .est field from Tab. \ref{Tab:Wbest}.\\[0.5ex]  \hline 
\texttt{Wstd}				 	& The standard deviation from all the bootstraps for all variabls in the .est field from Tab.\ref{Tab:Wbest}.\\[0.5ex]  \hline \hline
\texttt{WmeanN}				& Same as Wmean but for all model sizes. Computed if \texttt{fullBootstrap=true} was given in the runinput file (or chosen in the GUI).\\[0.5ex]  \hline 
\texttt{WstdN}					& Same as Wstd but for all model sizes. Computed if \texttt{fullBootstrap=true} was given in the runinput file (or chosen in the GUI).\\[0.5ex]  \hline 
\texttt{Fbootstrap}			& The score for each bootstrap and model size. Computed if \texttt{fullBootstrap=true} was given in the runinput file (or chosen in the GUI).\\[0.5ex]  \hline 
\texttt{pBest}					& The fraction of all bootstraps that resulted in the corresponding model sizes, \textit{i.e.} pBest(3)=0.8 means that 80\% of the bootstraps resulted in a 3 state model. Computed if \texttt{fullBootstrap=true} was given in the runinput file (or chosen in the GUI).\\[1ex] % [1ex] adds vertical space
\hline 
\end{tabular*}\label{Tab:bootstrap} 
\end{table}


\newpage
