In this section, we introduce the developed vbSPT software and how to use it.

\subsection{Recommended hardware}
It is computationally very demanding to identify the model that
describes thousands of trajectories the best. For this reason it is
strongly recommended that the analysis of data sets with more than
1000 trajectories are run on a very good computer. For example, a
typical analysis with 10000-15000 trajectories including bootstrapping
takes 2-10~h on a 2 x Intel� Xeon� X5650 (6 core, 2.66 GHz, 12MB L3)
machine running MATLAB verR2012a. It should be noted that this version
is capable of working with 12 parallel nodes, while previous versions
only work with 8, setting a limit on the actual number of usable cores
in the computer. However, the test example (see
Sec.~\ref{Sec:testRun}) should produce a result in less than 10~min
also on a conventional desktop or laptop computer.

\subsection{Installation}
To install the vbSPT software, uncompress the vbSPT.zip file into a
dedicated folder, which we will call \texttt{vbRoot/} in these
notes. The files under \mbox{\texttt{vbRoot/HMMcore/}} might need to
be recompiled, depending on what system you are running. To do this,
make sure you have a C compiler installed on you system and go to the
folder \texttt{vbRoot/HMMcore/} and run the script \texttt{compile\_code}. For a list of
compatible compilers see:\\
\mbox{http://www.mathworks.se/support/compilers/}. 

It is recommended to add the \texttt{vbRoot/} folder and the subfolders
\texttt{Tools}, \texttt{HMMcore/}, and \texttt{VB3/} to the Matlab
path. This can be done by executing the matlab script
\texttt{vbSPTstart} in the folder \texttt{vbRoot/}. If it is not added
to the Matlab path it has to be called from the same folder or with
its full path. Instructions for how to add these folders permanently to
your Matlab path can be found in the Matlab documentation.

It is also recommended to start using a static folder structure for
the analysis to keep it simple and benefit from the use of relative
paths (used in the GUI). The proposed structure consist of two
subfolders (\textit{e.g.} \texttt{InputData/} and \texttt{Results/})
located in a folder containing the runinputfiles that defines the
analysis parameters.

\subsection{Test runs}
\label{Sec:testRun}
In order to test if the installation is correct and working we have
included a small set of sample data and the corresponding
runinputfiles in the folder \texttt{vbRoot/ExampleData/}.

The example data set consists of 500 trajectories, with lengths chosen
from am exponential distribution with an average trajectory length of
10 positions, and a minimum of two positions. The data set was
generated by the script
\texttt{ExampleData/InputData/inputScript\_example.m}, with the following
parameters:
\begin{center}\begin{tabular}{ m{4cm} | >{\centering\arraybackslash}m{2.5cm} } 
\hline
\textbf{Parameter} & \textbf{Example data} \\ [0.5ex] 
\hline \hline 
$timestep$ [s]										&	  0.003\\ 
$P_1$										&	  0.67\\ 
$P_2$										&	  0.33\\ 
$D_1$			[\dc]						&	  1.0\\ 
$D_2$			[\dc]						&	  3.0\\ 
$A_{12}$	[timestep$^{-1}$]	&	  0.042\\ 
$A_{21}$	[timestep$^{-1}$]	&	  0.084\\ 
[1ex] % [1ex] adds vertical space
\hline 
\end{tabular}
\end{center}

To test if the software is working follow these steps:

\begin{itemize}
\item Add the above mentioned VB3 folders to your Matlab path,
  \textit{e.g.} by executing \texttt{vbSPTstart}.
\item Navigate Matlab to the \texttt{vbRoot/ExampleData/} folder.
\item Execute the command
  \verb+R=VB3_HMManalysis('runinput_short.m')+ in the Matlab
  prompt.
\end{itemize}

This should produce a file called
\texttt{testresult\_vbSPT\_HMM\_short.mat} in the
\texttt{vbRoot/Results/} subfolder, and also return the result in the
matlab struct \texttt{R}. If it does not work, disable parallel
computing by setting \texttt{parallelize\_config=false} in the
runinputfile and rerun the analysis. For a first look at the
results, execute the following:\\
\noindent \texttt{VB3\_getResult('runinput\_short.m')}.
Note that the diffusion constant is here given in units from the input data, so in this case [nm$^2$s$^{-1}$]. A graphical representation can be invoked by \texttt{VB3\_displayHMMmodel('runinput\_short.m')}. Here the diffusion constants are shown in [\dc], provided that the length and time units are given as in the GUI (see Section~\ref{Sec:gui}).

%\noindent \texttt{R.Wbest.est.DdtMean/3e-3/1e6} (diffusion constants, converted to [\dc]),\\
%\noindent \texttt{R.Wbest.est.Ptot} (occupation probabilities),\\
%\noindent \texttt{R.Wbest.est.Amean} (transition probability matrix).

It should be noted that the \texttt{runinput\_short.m} runinputfile
specifies an analysis of the data that actually ignores a large part
of the trajectories, by having a minimum trajectory length of 7.  A
larger analysis, that could take 20~min to 1~h on a laptop, can be
started by running \texttt{R=VB3\_HMManalysis('runinput\_normal.m')}. However, due to the small amount of data in the example data sets, one cannot expect
splendid numerical agreement with the input parameters.

\subsection{Analysis input}
The analysis takes two kinds of input:

\begin{itemize}
\item \textbf{runinputfile} - The file containing the parameters defining the input data as well as the analysis, see Section~\ref{Sec:runinput}.
\item \textbf{trajectories} - A .mat file containing at least one variable that is a cell array where each element, representing a trajectory, is a matrix where the rows define the coordinates in one, two or three dimensions in subsequent timesteps. The number of dimensions to be used for the analysis will be set by the runinputfile. 
\end{itemize}

The analysis is started either from the GUI or by the command \\ \texttt{VB3\_HMManalysis('runinputfilename')} in the Matlab prompt. 

\subsection{The runinput file}
\label{Sec:runinput}
At the center of the analysis is the runinputfile where the starting parameters are set. This file also acts as a handle for accessing the results and input data and can be used to do \textit{e.g.} extra bootstrapping analysis using the scripts presented in Section~\ref{Sec:usefulScripts}.

The runinputfile can be altered and modified by hand just as a text file or generated and edited through the graphical user interface (GUI). The parameters required in a runinputfile are listed and explained in Table~\ref{Tab:runinput}. 


\subsection{The graphical user interface (GUI)}
\label{Sec:gui}
The GUI is started from the Matlab prompt by the command \texttt{vbSPTgui}. It should be noted that within the GUI runinputfiles are referred to as scripts. From within the GUI it is possible to create new runinputfiles/scripts, load and edit as well as run them. It is also possible to print the result from a previous analysis in the Matlab prompt by loading its runinput file and choosing 'Show Result'. The GUI limits the input data to be in length units of either [nm] or [$\upmu$m] and time units to [s]. This is for the convenience of being able to supply initial guesses and results for diffusion coefficients in the common unit [\dc]. If other units are desired the user is limited to manually modifying the runinputfiles.


\subsection{Analysis results}
Here, we list the Matlab notation for some important variables contained within the result given by the analysis code. The analysis saves the result in a .mat file containing the following variables:

\begin{itemize}
\item \textbf{INF} - An array that documents the progress of the
  search algorithm, one converged model per row. Each row contains
  the \textbf{I}teration number (restart during which the model was generated), the \textbf{N}umber of states, and the lower bound on
  the evidence, \textbf{F}, which is the model score.
\item \textbf{Wbest} - A structure describing the best global model
  found by the analysis.
\item \textbf{WbestN} - A cell array containing structures that
  describes the best model for each model size as found by the
  analysis.
\item \textbf{bootstrap} - A structure containing the bootstrapping
  result for the best global model and also the best model for each
  model size provided that 'fullBootstrap=true' was given in the
  runinput file (or chosen in the GUI).
\item \textbf{dF} - An array showing the relative difference in the
  model score for different model sizes. The size for the best global
  model should have value 0, and all others negative.
\item \textbf{options} - A structure with fields defined by the
  runinput file that is used to run the analysis. 
\end{itemize}

In Table~\ref{Tab:Wbest} some important fields in the Wbest, and thus
also WbestN\{\textit{i}\} (where \textit{i} is a number describing the
model size) structure are presented and explained. All state-related
variables are sorted after increasing diffusion coefficient. In
Table~\ref{Tab:bootstrap} some important fields in the bootstrap
structure are presented and explained.

\subsection{Useful scripts}
\label{Sec:usefulScripts}
Here follows a brief description of some included useful scripts. For
further information on the input arguments and the scripts please
refer to the documentation in the .m files either by opening them or
running \texttt{help scriptname} from the Matlab prompt.

\begin{itemize}
\item \texttt{VB3\_getResult} - Loads the results from a previous
  analysis and prints some parameters in the Matlab prompt.
\item \texttt{VB3\_readData} - Loads the trajectory data set used for
  an analysis.
\item \texttt{VB3\_varyData} - Converges the best models for different
  model sizes with increasing amounts of input data.
\item \texttt{VB3\_generateSynthData} - Generates trajectories in a
  \textit{E.~coli} like geometry (tubular with spherical endcaps),
  according to model and geometry parameters specified by the
  user. Alternatively, the script used for generating the example data
  set (\texttt{inputScript\_example.m}, in the folder
  \texttt{vbRoot/ExampleData/InputData/}) can be modified and used to
  provide the input for this function.
\item \texttt{VB3\_bsResult} - Bootstraps the results of a finished
  analysis. Note that the bootstrapping parameters are taken from the
  the runinput file or options structure given as input, which should
  therefore be modified prior to running this script.
\item \texttt{VB3\_convertOld} - Converts models using direct
  transition probability parameterization (contains an \texttt{M.wA}
  field) to the present format.
\end{itemize}

Additional undocumented scripts that can serve as templates for visualizing and extracting data can be found under \texttt{vbRoot/Tools/} and subfolders therein.

\subsection{A note on units}
\label{Sec:units}
The testdata as well as the real data analyzed in the original paper,
measured lengths in nanometers and time in seconds, and hence we divide
diffusion constants by $10^6$ to convert them to the more convenient
units of \dc. The GUI allows the user some choice in specifying what
units to use, and then writes the runinputfile accordingly.

The analysis software in itself does not know about units however, and
will analyze data in arbitrary units in a consistent manner.  The unit
of time is specified in the \textbf{timestep} parameter of the
runinput file (see next section), while the length unit is set by the
data. Hence, the user must make sure that diffusion constants in the
runinputfile (for initial guesses and priors) are given in the same
unit system.
 
\subsection{More information}
For more information about any .m file, run \texttt{help filename} in
the Matlab prompt, where filename should start with 'VB3\_'.
