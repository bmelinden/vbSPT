In this section, we introduce the developed vbSPT software and how to use it.

\subsection{Recommended hardware}
It is computationally very demanding to identify the model that describes thousands of trajectories the best. For this reason it is strongly recommended  that the analysis of data sets with more than 1000 trajectories are run on a very good computer. For example, a typical analysis with 10000-15000 trajectories including bootstrapping takes 2-10~h on a 2 x Intel� Xeon� X5650 (6 core, 2.66 GHz, 12MB L3) machine running MATLAB verR2012a. It should be noted that this version is capable of working with 12 parallel nodes, while previous versions only work with 8, setting a limit on the actual number of usable cores in the computer. However, the test example (see Sec.~\ref{Sec:testRun}) should produce a result in less than 10~min also on a conventional desktop or laptop computer. 

\subsection{Installation}
To install the vbSPT software, uncompress the vbSPT.zip file into a dedicated folder. The files under 'HMMcode/HMMcore' might need to be recompiled, depending on what system you are running. To do this, make sure you have a compiler installed on you system (for a list of compatible compilers see http://www.mathworks.se/support/compilers/). Then go to the folder 'HMMcore' and run the script \texttt{compile\_code}.

It is recommended to add the folder ('HMMcode'), and the subfolders 'Tools', 'HMMcore', and 'VB3' to the Matlab path. This can be done by executing the script 'VB3start.m' present in the 'HMMcode' folder. If it is not added to the Matlab path it has to be called from the same folder or with its path. Instructions for how to add these folders permanently to your Matlab path can be found in the Matlab documentation.

It is also recommended to start using a static folder structure for the analysis to keep it simple and benefit from the use of relative paths (used in the GUI). The proposed structure consist of two subfolders (\textit{e.g.} 'InputData' and 'Results') located in a folder containing the runinputfiles that defines the analysis parameters.

\subsection{Test runs}
\label{Sec:testRun}
In order to test if the installation is correct and working we have included a small set of sample data and the corresponding runinputfiles in the folder 'VB3\_example'. 

The example data set consists of 500 trajectories chosen from a normal distribution with an average trajectory length of 10 jumps and standard deviation of 5. The data set was generated by the script \texttt{inputScript\_example.m} present in the 'InputData' folder with the following parameters:

\begin{center}\begin{tabular}{ m{4cm} | >{\centering\arraybackslash}m{2.5cm} } 
\hline
\textbf{Parameter} & \textbf{Example data} \\ [0.5ex] 
\hline \hline 
$P_1$										&	  0.67\\ 
$P_2$										&	  0.33\\ 
$D_1$			[\dc]						&	  1.0\\ 
$D_2$			[\dc]						&	  3.0\\ 
$A_{12}$	[timestep$^{-1}$]	&	  0.045\\ 
$A_{21}$	[timestep$^{-1}$]	&	  0.090\\ 
[1ex] % [1ex] adds vertical space
\hline 
\end{tabular}
\end{center}

To test if the software is working follow these steps:

\begin{itemize}
\item Add the above mentioned VB3 folders to your matlab path, \textit{e.g.} by executing \texttt{VB3start}.
\item Navigate Matlab to the 'VB3\_example' folder.
\item Execute the command
  \texttt{VB3\_HMManalysis('runinput\_short.m')} in the Matlab
  prompt.
\end{itemize}

This should produce a file called 'testresult\_VB3\_HMM\_short.mat' in the 'Results' subfolder. If it does not work, disable parallel computing by setting \texttt{parallelize\_config=false} in the runinputfile and rerun the analysis.
It should be noted that the 'runinput\_short' file only performs a short analysis of the data and actually ignores a large part of the trajectories (by having a minimum trajectory length of 7 jumps). For a more thorough analysis run \texttt{VB3\_HMManalysis('runinput.m')} (Note that this could potentially take between 30~min and 1~h on a laptop). However, due to the small amount of data in the example data set it should not be expected to improve the result drastically. 

\subsection{Analysis input}
The analysis takes two kinds of input:

\begin{itemize}
\item \textbf{runinputfile} - The file containing the parameters defining the input data as well as the analysis, see Section~\ref{Sec:runinput}.
\item \textbf{trajectories} - A .mat file containing at least one variable that is a cell array where each element, representing a trajectory, is a matrix where the rows define the coordinates in one, two or three dimensions in subsequent timesteps. The number of dimensions to be used for the analysis will be set by the runinputfile. 
\end{itemize}

The analysis is started either from the GUI or by the command \\ \texttt{VB3\_HMManalysis('runinputfilename')} in the Matlab prompt. 

\subsection{The runinput file}
\label{Sec:runinput}
At the center of the analysis is the runinputfile where the starting parameters are set. This file also acts as a handle for accessing the results and input data and can be used to do \textit{e.g.} extra bootstrapping analysis using the scripts presented in Section~\ref{Sec:usefulScripts}.

The runinputfile can be altered and modified by hand just as a text file or generated and edited through the graphical user interface (GUI). The parameters required in a runinputfile are listed and explained in Table~\ref{Tab:runinput}. 


\subsection{The graphical user interface (GUI)}
The GUI is started from the Matlab prompt by the command \texttt{vbSPTgui}. From within the GUI it is possible to create new runinputfiles, load and edit as well as run the runinputfiles. It is also possible to print the result from a previous analysis in the Matlab prompt by loading its runinput file and choosing 'Show Result'. It should be noted that within the GUI runinputfiles are referred to as scripts. 


\subsection{Analysis results}
Here, we list the Matlab notation for some important variables contained within the result given by the analysis code. The analysis saves the result in a .mat file containing the following variables:

\begin{itemize}
\item \textbf{INF} - An array that documents the progress of the
  search algorithm, one converged model per row. Each row contains
  the \textbf{I}teration number (restart during which the model was generated), the \textbf{N}umber of states, and the lower bound on
  the evidence, \textbf{F}, which is the model score.
\item \textbf{Wbest} - A structure describing the best global model
  found by the analysis.
\item \textbf{WbestN} - A cell array containing structures that
  describes the best model for each model size as found by the
  analysis.
\item \textbf{bootstrap} - A structure containing the bootstrapping
  result for the best global model and also the best model for each
  model size provided that 'fullBootstrap=true' was given in the
  runinput file (or chosen in the GUI).
\item \textbf{dF} - An array showing the relative difference in the
  model score for different model sizes. The size for the best global
  model should have value 0.
\item \textbf{options} - A structure with fields defined by the
  runinput file that is used to run the analysis. 
\end{itemize}

In Table~\ref{Tab:Wbest} some important fields in the Wbest, and thus also WbestN\{\textit{i}\} (where \textit{i} is a number describing the model size) structure are presented and explained. All state-related variables are sorted after increasing diffusion coefficient. In Table~\ref{Tab:bootstrap} some important fields in the bootstrap structure are presented and explained.

\subsection{Useful scripts}
\label{Sec:usefulScripts}
Here follows a brief description of some included useful scripts. For further information on the input arguments and the scripts please refer to the documentation in the .m files either by opening them or running \texttt{help scriptname} in the Matlab prompt.

\begin{itemize}
\item \texttt{VB3\_getResult} - Loads the results from a previous analysis and prints some parameters in the Matlab prompt. It takes a runinputfile as input argument.
\item \texttt{VB3\_readData} - Loads the trajectory data set used for an analysis. It takes a runinputfile as input argument.
\item \texttt{VB3\_varyData} - Converges the best models for different model sizes with increasing amounts of input data, which can be specified as an option. It takes a runinputfile and options as input arguments. 
\item \texttt{VB3\_generateSynthData} - Generates trajectories in a \textit{E.~coli} like geometry (tubular with spherical endcaps). All parameters can be set by using options. If a runinputfile is included as the first argument then all undefined options are taken from a previous analysis result defined by the runinputfile. Alternatively the script used for generating the example data set (\texttt{inputScript\_example.m} in the folder 'VB3\_example/InputData') can be modified and used to provide the input for this function. 
\item \texttt{VB3\_bsResult} - Bootstraps a finished result based on the bootstrapping parameters given in the runinputfile. These should be modified prior to running this script. It takes a runinputfile and options as input arguments.
\end{itemize}

\subsection{More information}
For more information about any .m file, run \texttt{help filename} in the Matlab prompt, where filename should start with 'VB3\_'.
